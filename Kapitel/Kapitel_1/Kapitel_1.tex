\documentclass[
    12pt,
    a4paper,
    ngerman,
    color=3b,% Farbe für Hervorhebungen auf Basis der Deklarationen in den
    %type=intern,
    %titlepage=true,
    marginpar=false,
    colorback=false,
    %logo=head,
    leqno,
]{tudaexercise}
\usepackage{import}
% Import all Packages from Main Preamble with relative Path
\subimport*{../../}{preamble}
% Get Labels from Main Document using the xr-hyper Package
\externaldocument{../../AuD-Zusammenfassung-2020}
\externaldocument{../Kapitel_2/Kapitel_2}
% Set Graphics Path, so pictures load correctly
\graphicspath{{../../}}

\begin{document}
\section{Einführung}\label{1}\label{Einfuehrung}
\subsection{Probleme in der Informatik}\label{1.1}
Ein \fatsf{Problem} im Sinne der Informatik:
\begin{itemize}
    \item Enthält Beschreibung der Eingabe
    \item Enthält Beschreibung der Ausgabe
    \item Gibt selbst \fatsf{keinen} Übergang von Ein und Ausgabe an
\end{itemize}
\begin{figure}[ht]
    \centering
    \includestandalone[width=.5\textwidth]{pictures/problem_informatik_modell/problem_informatik_modell}% 
    \caption{Modell Problem Informatik}
    \label{fig:modell-problem-informatik}
\end{figure}
z.B. Finde den kürzesten Weg zwischen 2 Orten\\
Eine \fatsf{Probleminstanz} ist eine konkrete Eingabebelegung für die entsprechende Ausgabe gewünscht.\\
Für das obige Problem wäre das z.B. "`Was ist der kürzeste Weg vom Audimax in die Mensa?"'
\subsection{Definitionen für Algorithmen}\label{1.2}\label{Definitionen fuer Algorithmen}
"`Ein Algorithmus ist eine \fatsf{endliche Folge} von Rechenschritten, die eine \fatsf{Eingabe} in eine \fatsf{Ausgabe} umwandelt."'\footnote{Quelle: Cormen et al., 4. Auflage}\\
Anforderungen an Algorithmen:\begin{itemize}
    \item Spezifizierung der Ein- und Ausgabe:\begin{itemize}
              \item Anzahl und Typen aller Elemente ist/sind definiert
          \end{itemize}
    \item Eindeutigkeit:\begin{itemize}
              \item Jeder Einzelschritt ist klar definiert und ausführbar
              \item Die Reihenfolge der Einzelschritte ist festgelegt.
          \end{itemize}
    \item Endlichkeit\begin{itemize}
              \item Notation hat endliche Länge
          \end{itemize}
\end{itemize}
Eigenschaften von Algorithmen:\begin{itemize}
    \item Determiniertheit:\begin{itemize}
              \item Für gleiche Eingabe folgt stets die gleiche Ausgabe (andere Zwischenzustände sind möglich)
          \end{itemize}
    \item Determinismus:\begin{itemize}
              \item Für die gleiche Eingabe ist die Ausführung und Ausgabe stets identisch.
          \end{itemize}
    \item Terminierung:\begin{itemize}
              \item Der Algorithmus läuft für jede endliche Eingabe nur endlich lange
          \end{itemize}
    \item Korrektheit:\begin{itemize}
              \item Der Algorithmus berechnet stets die spezifizierte Ausgabe (falls dieser
                    terminiert).
          \end{itemize}
    \item Effizienz:\begin{itemize}
              \item Sparsamkeit im Ressourcenverbrauch (Zeit, Speicher, Energie, ...)
          \end{itemize}
\end{itemize}
\subsection{Definitionen für Datenstrukturen}\label{1.3}\label{Definitionen fuer Datenstrukturen}
"`Eine Datenstruktur ist eine Methode,
Daten \fatsf{abzuspeichern} und zu \fatsf{organisieren} sowie
den \fatsf{Zugriff} auf die Daten und die \fatsf{Modifikation}
der Daten zu erleichtern."'\footnote{Quelle: Cormen et al., 4. Auflage}\\
\begin{wrapfigure}[5]{r}{.6\textwidth}
    \centering
    \includestandalone[width=.5\textwidth]{pictures/baum_beispiel/baum_beispiel}% 
    \caption{Beispiel Datenstruktur (Rot-Schwarz-Baum)}
    \label{fig:baum_beispiel}
\end{wrapfigure}
Datenstrukturen:\begin{itemize}
    \item Sind Organisationsformen für Daten
    \item Beinhalten Strukturbestandteile und Nutzerdaten (Payload)
\end{itemize}
z.B. \hyperref[2.2]{Arrays}, listen, \ldots
\vspace*{2cm}
\end{document}