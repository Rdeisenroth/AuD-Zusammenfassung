\documentclass[
    ngerman,
    color=3b,
    dark_mode,
    load_common, % Loads a list of commonly used Packages
    summary,
    boxarc,
    % manual_term,
    % solution=true,
]{rubos-tuda-template} 
% Import all Packages from Main Preamble with relative Path
% \subimport*{../../}{preamble}
% Get Labels from Main Document using the xr-hyper Package
\externaldocument{../../AuD-Zusammenfassung-2020}
% Set Graphics Path, so pictures load correctly
\graphicspath{{../../}}

\begin{document}
\section{Graph Algorithms}\index{Graph Algorithms}
\subsection{Graphen}\index{Graphen}
\begin{definition}[(Endlicher) gerichteter Graph]\mbox{}
    \begin{itemize}
        \item (endlicher) gerichteter Graph $G = (V,E)$
        \item besteht aus (endlicher) Knotenmenge $V$
        \item besteht aus (endlicher) Kantenmenge $E \subseteq V x V$
        \item $(u,v) \in E$: Kanten von Knoten $u$ zu $v$
        \item Kanten haben eine Richtung
    \end{itemize}
\end{definition}

\begin{definition}[Ungerichteter Graph]\mbox{}
    \begin{itemize}
        \item (endlicher) ungerichteter Graph $G = (V,E)$
        \item besteht aus (endlicher) Knotenmenge $V$
        \item besteht aus (endlicher) Kantenmenge $E \subseteq V x V$, sodass $(u,v) \in E \Leftrightarrow (v,u) \in E$
        \item Kanten haben keine Richtung
    \end{itemize}
\end{definition}

\paragraph{Darstellung von Graphen}
\begin{itemize}
    \item Als Adjazentmatrix\index{Adjazenzmatrix} (1, wenn Kante von i zu j bzw. 0, wenn keine Kante)
    \item Bei ungerichteten Graphen ist Matrix spiegelsymmetrisch zur Hauptdiagonalen
    \item[$\Longrightarrow$] Speicherbedarf: $\Theta(|V^2|)$
\end{itemize}
\begin{figure}[h]
    \centering
    \begin{subfigure}[c]{.5\textwidth}
        \centering
        \includegraphics[width=.5\textwidth]{pictures/graph1\IfDarkModeT{_dark}.PNG}
        \caption{Darstellung als Adjazentmatrix}
    \end{subfigure}
    \begin{subfigure}[c]{.49\textwidth}
        \centering
        \includegraphics[width=.5\textwidth]{pictures/graph2\IfDarkModeT{_dark}.PNG}
        \caption{Grafische Darstellung}
    \end{subfigure}
    \caption{Beispielhafte Darstellung eines Graphen}
\end{figure}
% \begin{minipage}{0.45\textwidth}
%     \includegraphics[width=4cm]{pictures/graph1.PNG}
% \end{minipage}
% \begin{minipage}{0.45\textwidth}
%     \includegraphics[width=4cm]{pictures/graph2.PNG}
% \end{minipage}
\begin{itemize}
    \item Auch darstellbar als Array mit verketteten Listen
    \item[$\Longrightarrow$] Speicherbedarf: $\Theta(|V| + |E|)$
\end{itemize}
\clearpage

\paragraph{Pfadfinder}\index{Pfadfinder}\mbox{}
\begin{grayInfoBox}
    \begin{itemize}
        \item Knoten $v$ ist von Knoten $u$ erreichbar, wenn es wenn es von $u$ aus einen Pfad über $n$ Knoten nach $v$
        \item $u$ ist immer von $u$ per leerem Pfad (k=1) erreichbar
        \item Länge des Pfades = $k - 1$ = Anzahl Kanten
    \end{itemize}
\end{grayInfoBox}

\begin{definition}[Zusammenhängende Graphen]\index{Zusammenhänge}\mbox{}
    \begin{itemize}
        \item Ungerichtet: Zusammenhängend\index{Zusammenhänge!zusammenhängend} wenn jeder Knoten von jedem anderen Knoten aus erreichbar ist
        \item Gerichtet: \fatsf{Stark}\index{Zusammenhänge!stark zusammenhängend} zusammenhängend, wenn obiges auch gemä\ss{} Kantenrichtung gilt
    \end{itemize}
\end{definition}

\paragraph{Bäume und Subgraphen}\mbox{}
\begin{grayInfoBox}
    \begin{itemize}
        \item Graph $G$ ist ein Baum, wenn $V$ leer ist oder wenn es einen Knoten in $V$ gibt,
              von dem aus jeder andere Knoten eindeutig erreichbar ist (Wurzel).
        \item Graph $G'=(V',E')$ ist Subgraph\index{Subgraph} von $G=(V,E)$, wenn $V'\subseteq V$ und $E' \subseteq E$.
    \end{itemize}
\end{grayInfoBox}

\begin{definition}[Gewichtete Graphen]\index{Gewichtete Graphen}\mbox{}
    \begin{itemize}
        \item gewichteter gerichteter oder ungerichteter Graph $G=(V,E)$
        \item besitzt zusätzlich Funktion $w: E \rightarrow R$
        \item Angabe des Gewichts einer Kante $u$ nach $v$ durch $w((u,v))$
    \end{itemize}
\end{definition}
\clearpage
\subsection{Breadth-First Search (BFS)}\index{Breadth-First Search (BFS)}
\begin{idea}[Breadth-First Search]\mbox{}
    \begin{itemize}
        \item Besuche zuerst alle unmittelbaren Nachbarn, dann deren Nachbarn, usw.
        \item Anwendung: Webcrawling, Garbage Collection,..
    \end{itemize}
\end{idea}

\paragraph{Algorithmus}\mbox{}
%\begin{noindent}
\begin{codeBlock}[autogobble,escapeinside=||]{title={BFS(G,s) //G=(V,E) s = source node in V}}
    BFS(G,s) //G=(V,E) s = source node in V
    FOREACH u in V-{s} DO
        u.color = WHITE;        // Wei\ss{} = noch nicht besucht
        u.dist = |$+\infty$|            // Setzen der Distanzen auf Unendlich
        u.pred = nil;           // Setzen der Vorgänger auf nil
    s.color = GRAY;             // Anfang bei Startnode
    s.dist = 0;
    s.pred = nil;
    newQueue(Q);
    enqueue(Q,s);
    WHILE !isEmpty(Q) DO
        u = dequeue(Q); 
        FOREACH v in adj(G,u) DO
            IF v.color == WHITE THEN
                v.color == GRAY;
                v.dist = u.dist+1;
                v.pred = u;
                enqueue(Q,v);
        u.color = BLACK;            // Knoten abgearbeitet
\end{codeBlock}
%\end{noindent}
\begin{tcolorbox}[
        colback=yellow!20,
        colframe=black!50!red!100,
        enhanced,
        title={Farben:}
    ]
    \begin{itemize}
        \item {\makebox[\widthof{\texttt{WHITE}:}][l]{\texttt{WHITE}:} Knoten noch nicht besucht}
        \item {\makebox[\widthof{\texttt{WHITE}:}][l]{\texttt{GRAY}:} Knoten in Queue für nächsten Schritt}
        \item {\makebox[\widthof{\texttt{WHITE}:}][l]{\texttt{BLACK}:} Knoten ist fertig}
    \end{itemize}
\end{tcolorbox}
\begin{itemize}
    \item Laufzeit: $O(|V| + |E|)$
    \item Nach Algorithmus steht in $v$ die kürzeste Distanz von $s$ nach $v$
\end{itemize}

\pagebreak

\paragraph{Kürzeste Pfade ausgeben}\mbox{}
%\begin{noindent}
\begin{codeBlock}[autogobble]{title={print-path(G,s,v) // Assumes that BFS(G,s) has already been executed}}
IF v == s THEN
    print s;
ELSE
    IF v.pred == nil THEN
        print "no path from s to v"
    ELSE
        print-path(G,s,v.pred);
        print v;
\end{codeBlock}
%\end{noindent}

\paragraph{Abgeleiteter BFS-Baum}\mbox{}
\begin{figure}[h]
    \centering
    \includegraphics[width=14cm]{pictures/bfsBaum\IfDarkModeT{_dark}.PNG}
    \caption{Beispiel BFS-Baum}
\end{figure}
\begin{itemize}
    \item Subgraph $G^s_{pred} = (V^s_{pred},E^s_{pred})$ von G:
          \begin{itemize}
              \item $V^s_{pred} = \{v \in V | v.pred \neq nil\} \cup \{s\}$
              \item $E^s_{pred} = \{(v.pred,v) | v \in V^s_{pred} - \{s\}\}$
          \end{itemize}
    \item $G^s_{pred}$ enthält alle von $s$ aus erreichbaren Knoten in $G$
    \item Au\ss{}erdem handelt es sich hier nur um kürzeste Pfade
\end{itemize}
\clearpage

\subsection{Depth-First Search(DFS)}\index{Depth-First Search(DFS)}
\begin{idea}[Depth-First Search]
    \begin{itemize}
        \item Besuche zuerst alle noch nicht besuchten Nachfolgeknoten
        \item \string"Laufe so weit wie möglich weg vom aktuellen Knoten\string"
    \end{itemize}
\end{idea}
\paragraph{Algorithmus}\mbox{}
%\begin{noindent}
\begin{codeBlock}[autogobble]{title={ DFS(G)}}
FOREACH u in V DO
    u.color = WHITE;
    u.pred = nil;
time = 0;               // time hier als globale Variable
FOREACH u in v DO
    IF u.color == WHITE THEN
        DFS-VISIT(G,u)  // Start eines rekursiven Aufrufs
\end{codeBlock}
%\end{noindent}
%\begin{noindent}
\begin{codeBlock}[autogobble]{title={DFS-VISIT(G,u)}}
time = time + 1;
u.disc = time;          // discovery time
u.color = GRAY;
FOREACH v in adj(G,u) DO
    IF v.color == WHITE THEN
        v.pred = u;
        DFS-VISIT(G,v);
u.color = BLACK;
time = time + 1;
u.finish = time;        // finish time
\end{codeBlock}
%\end{noindent}

\paragraph{DFS-Wald = Menge von DFS-Bäumen}\index{DFS-Wald}\mbox{}\vspace{-2em}\\
\begin{wrapfigure}[4]{r}{9cm}
    \centering
    % trim={<left> <lower> <right> <upper>}
    \includegraphics[width=9cm,trim={0 2.2cm 0 0},clip]{pictures/dfswald\IfDarkModeT{_dark}.PNG}
    \captionof{figure}{Beispiel DFS-Wald}
\end{wrapfigure}
\begin{itemize}
    \item Subgraph $G_{pred}=(V,E_{pred})$ von G
    \item besteht aus $E_{pred} = {(v.pred,v)|v \in V, v.pred \neq nil}$
    \item DFS-Baum gibt nicht unbedingt den kürzesten Weg wieder
\end{itemize}
\vspace*{1cm}
\paragraph{Kantenarten}
\begin{description}[leftmargin=3cm]
    \item [Baumkanten] alle Kanten in $G_{pred}$
    \item [Vorwärtskanten] alle Kanten in $G$ zu Nachkommen in $G_{pred}$, die keine Baumkante sind
    \item [Rückwärtskanten] alle Kanten in $G$ zu Vorfahren in $G_{pred}$, die keine Baumkante sind (inkl. Schleifen)
    \item [Kreuzkanten] alle anderen Kanten in $G$
\end{description}

\paragraph{Anwendungen DFS}
\begin{wrapfigure}[5]{r}{8cm}
    \centering
    \includegraphics[width=8cm]{pictures/topo\IfDarkModeT{_dark}.PNG}
    \caption{Beispiel Topoligisches Sortieren}
\end{wrapfigure}
\begin{itemize}
    \item Job Scheduling (Job X muss vor Job Y beendet sein)
    \item Topologisches Sortieren
          \begin{itemize}
              \item nur für dag (directed acyclic graph)
              \item Kanten immer nur nach rechts
              \item Sortierung aber nicht eindeutig
          \end{itemize}
\end{itemize}

%\begin{noindent}
\begin{codeBlock}[autogobble]{title={TOPOLOGICAL-SORT(G)}}
    new LinkedList(L);
    run DFS(G) but, each time a node is finished, insert in front of L
    return L.head;
\end{codeBlock}
%\end{noindent}

\paragraph{Starke Zusammenhangskomponenten}\index{Starke Zusammenhangskomponenten}\mbox{}\vspace{-1em}\\
\begin{wrapfigure}[5]{r}{6cm}
    \centering
    \includegraphics[width=6cm]{pictures/scc\IfDarkModeT{_dark}.PNG}
    \captionof{figure}{Beispiel Starke Zusammenhangskomponenten}
\end{wrapfigure}
Knotenmenge $C \subseteq V$, so dass es zwischen zwei Knoten $u,v \in C$ einen Pfad von $u$ nach $v$ gibt und es keine Menge $D \subseteq V$ mit $C \subsetneq D$ gibt, für die obiges auch gilt.\\
Eigenschaften:
\begin{itemize}
    \item Verschiedene SCC's sind disjunkt
    \item Zwei SCC's sind nur in eine Richtung verbunden
\end{itemize}
Algorithmus:\\
DFS zweimal laufen lassen
\begin{itemize}
    \item Einmal auf Graph $G$
    \item Einmal auf Graph $G^T = (V,E^T)$ (transponiert)
\end{itemize}
Dadurch bleiben die SCC's gleich, die Kanten drehen sich aber jeweils um\vspace{2em}\\
Code:
%\begin{noindent}
\begin{codeBlock}[autogobble,escapeinside=||]{title={SCC(G)}}
run DFS(G)
compute |$G^T$|
run DFS(|$G^T$|) but visit vertices in main loop
    in descending finish time from 1
output each DFS tree from above as one SCC
\end{codeBlock}
%\end{noindent}

\clearpage
\subsection{Minimale Spannbäume}\index{Minimale Spannbäume}
\begin{definition}[Minimaler Spannbaum]\mbox{}
    \begin{itemize}
        \item Verbindung aller Knoten miteinander
        \item Minimaler Spannbaum $\Rightarrow$ Minimales Gewicht
    \end{itemize}
\end{definition}

\paragraph{Allgemeiner Algorithmus}\mbox{}
%\begin{noindent}
\begin{codeBlock}[autogobble,escapeinside=||]{title={genericMST(G,w)}}
A = |$\emptyset$|
WHILE A does not form a spanning tree for G DO
    find safe edge {u,v} for A
    A = A |$\cup$|{{u,v}}
return A
\end{codeBlock}
%\end{noindent}
\begin{minipage}{0.35\textwidth}
    \captionsetup{type=figure}
    \includegraphics[width=6cm]{pictures/mstTerm\IfDarkModeT{_dark}.PNG}
    \captionof{figure}{Beispiel Ausführung genericMST()}
\end{minipage}
\begin{minipage}{0.55\textwidth}
    Terminologie:
    \begin{itemize}
        \item Schnitt (S, V-S) partioniert Knoten in zwei Mengen
        \item \{u,v\} überbrückt Schnitt, wenn $u \in S$ und $v \in V-S$
        \item Schnitt respektiert $A \subseteq E$, wenn keine Kante \{u,v\} aus A den Schnitt überbrückt
        \item \{u,v\} leichte Kante für (S, V-S), wenn w(\{u,v\}) minimal für alle den Schnitt überbrückenden Kanten
        \item \{u,v\} sicher für A, wenn $A \cup \{\{u,v\}\}$ Teilmenge eines MST
    \end{itemize}
\end{minipage}

\paragraph{Algorithmus von Kruskal}\index{Kruskal}\mbox{}

\begin{idea}[Algorithmus von Kruskal]\mbox{}
    Suchen der \enquote{kleinsten} Kante und Zusammenfügen von Mengen, falls Mengen ungleich sind
\end{idea}
\begin{itemize}
    \item Lässt parallel mehrere Unterbäume eines MST wachsen
    \item Laufzeit: $O(|E| \cdot log|E|)$
\end{itemize}
%\begin{noindent}
\begin{codeBlock}[autogobble,escapeinside=||,fontsize=\small]{title={MST-Kruskal(G,w)}}
A = |$\emptyset$|
FOREACH v in V DO
    set(v) = {v};       // Menge mit sich selbst
Sort edges according to weight in nondecreasing order
FOREACH {u,v} in E according to order DO
    IF set(u) != set(v) THEN    // Mengen noch nicht verbunden
        A = A |$\cup$| {{u,v}}; 
        UNION(G,u,v);   // Zusammenführen der Mengen aller Knoten aus den Sets
return A;
\end{codeBlock}
%\end{noindent}
\clearpage
\paragraph{Algorithmus von Prim}\index{Prim}
\begin{itemize}
    \item Konstruiert einen MST Knoten für Knoten
    \item Fügt immer leichte Kante zu zusammenhängender Menge hinzu
    \item Laufzeit: $O(|E| + |V| \cdot log|V|)$
\end{itemize}
%\begin{noindent}
\begin{codeBlock}[autogobble,escapeinside=||]{title={MST-Prim(G,w,r) // r is given root}}
    FOREACH v in V DO 
        v.key = +|$\infty$|;
        v.pred = nil;
    r.key = -|$\infty$|
    Q = V;
    WHILE !isEmpty(Q) DO
        u = EXTRACT-MIN(Q);     // smallest key value
        FOREACH v in adj(u) DO
            IF v|$\in$|Q and w({u,v})<v.key THEN
                v.key = w({u,v});
                v.pred = u;
\end{codeBlock}
%\end{noindent}


\subsection{Kürzeste Wege in (gerichteten) Graphen}\index{Kürzeste Wege}
\begin{definition}[SSSP]\mbox{}
    \begin{itemize}
        \item SSSP - Single-Source Shortest Path\index{SSSP}
        \item Von Quelle $s$ ausgehend die kürzesten Pfad zu allen anderen Knoten
        \item Kürzester Pfad: Pfad mit minimalem Gesamtgewicht von einem zum anderen Knoten
        \item BFS findet nur minimale Kantenwege (nicht Gewichtswege)
        \item MST minimiert das Gesamtgewicht des Baumes (nicht zu einzelnen Kanten)
        \item Negative Kantengewichte sind erlaubt, aber keine Zyklen mit negativem Gesamtgewicht
    \end{itemize}
\end{definition}
\vspace{-1em}
\paragraph{Gemeinsame Idee für Algorithmen - Relax}\index{Relax}\mbox{}\\
Verringere aktuelle Distanz von Knoten $v$, wenn durch Kante $(u,v)$ kürzer erreichbar\\
\includegraphics[width=12cm]{pictures/ssspRelax\IfDarkModeT{_dark}.PNG}
%\begin{noindent}
\begin{codeBlock}[autogobble]{title={relax(G,u,v,w)}}
IF v.dist > u.dist + w((u,v)) THEN
    v.dist = u.dist + w((u,v));
    v.pred = u;
\end{codeBlock}
%\end{noindent}
\clearpage
\paragraph{Bellman-Ford-Algorithmus}\index{Bellman-Ford}\mbox{}\\
Laufzeit: $\Theta(|E| \cdot |V|)$
%\begin{noindent}
\begin{codeBlock}[autogobble,escapeinside=||]{title={Bellman-Ford-SSSP(G,s,w)}}
    initSSSP(G,s,w);
    FOR i = 1 TO |$\vert V\vert$|-1 DO
        FOREACH (u,v) in E DO
            relax(G,u,v,w);
    FOREACH (u,v) in E DO   // Prüfung ob negativer Zyklus
        IF v.dist > u.dist+w((u,v)) THEN
            return false;
    return true;
\end{codeBlock}
%\end{noindent}
%\begin{noindent}
\begin{codeBlock}[autogobble,escapeinside=||]{title={initSSSP(G,s,w)}}
    FOREACH v in V DO
        v.dist = |$\infty$|;
        v.pred = nil;
    s.dist = 0;
\end{codeBlock}
%\end{noindent}
\vspace{-1em}
\paragraph{TopoSort für dag}\index{TopoSort}\mbox{}
\begin{idea}[TopoSort für dag]
    Erhalten des kürzesten Pfades durch das topologische Sortieren
\end{idea}
Laufzeit: $\Theta(|E| + |V|)$

%\begin{noindent}
\begin{codeBlock}[autogobble]{title={TopoSort-SSSP(G,s,w)    // G muss dag sein}}
    initSSSP(G,s,w);
    execute topological sorting
    FOREACH u in V in topological order DO
        FOREACH v in adj(u) DO
            relax(G,u,v,w);
\end{codeBlock}
%\end{noindent}

\vspace{-1em}
\paragraph{Dijkstra-Algorithmus}\index{Dijkstra}\mbox{}\\
Voraussetzung: Keine negativen Kantengewichte\\
Laufzeit: $\Theta(|V| \cdot log|V| + |E|)$
%\begin{noindent}
\begin{codeBlock}[autogobble]{title={Dijkstra-SSSP(G,s,w)}}
    initSSSP(G,s,w);
    Q = V;
    WHILE !isEmpty(Q) DO
        u = EXTRACT-MIN(Q);     // smallest distance
        FOREACH v in adj(u) DO
            relax(G,u,v,w);
\end{codeBlock}
%\end{noindent}
\begin{minipage}{.4\textwidth}
    \includestandalone[]{pictures/dijkstra-fail-graph/dijkstra-fail-graph}
\end{minipage}%
\begin{minipage}{.6\textwidth}
    \begin{itemize}
        \item Beispiel für Problem mit negativen Kantengewisten bei Dijkstra: Dijkstra würde Pfad 1-2-3 liefern, da das Kantengewicht 4 grö\ss{}er als der andere Pfad ist.
    \end{itemize}
\end{minipage}

\clearpage
\subsection{Maximaler Fluss in Graphen}\index{Maximaler Fluss}\mbox{}\vspace{-2em}
\begin{wrapfigure}[2]{r}{6cm}
    \includegraphics[width=6cm]{pictures/flussIdee\IfDarkModeT{_dark}.PNG}
    \caption{Beispiel Flussnetzwerk}
\end{wrapfigure}
\begin{idea}[Maximaler Fluss im Graphen]\mbox{}
    \begin{itemize}
        \item Kanten haben Flusswert und maximale Kapazität
        \item Jeder Knoten (au\ss{}er s und t) haben den gleichen eingehenden und ausgehenden Fluss
        \item Ziel: Finde maximalen Fluss von s nach t
        \item s: Source/Quelle
        \item t: Target/Senke
    \end{itemize}
\end{idea}

\begin{definition}[Flussnetzwerk]
    Ein Flussnetzwerk\index{Flussnetzwerk} ist ein gewichteter, gerichteter Graph $G=(V,E)$ mit Kapazität $c$, so dass
    $c(u,v) \geq 0$ für $(u,v) \in E$ und $c(u,v) = 0$ für $(u,v) \notin E$, mit zwei Knoten $s,t \in V$,
    so dass jeder Knoten von $s$ aus erreichbar ist und $t$ von jedem Knoten aus erreichbar ist.
    Damit gilt $|E| \geq |V| - 1$.
\end{definition}
\begin{definition}[Fluss]\index{Fluss}\mbox{}\\
    Ein Fluss $f: V x V \rightarrow \mathbb{R}$ für ein Flussnetzwerk $G = (V,E)$ mit Kapazität $c$
    und Quelle $s$ und Senke $t$ erfüllt $0 \leq f(u,v) \leq c(u,v)$ für alle $u,v \in V$, sowie für alle
    $u \in V - \{s,t\}$: \\
    $\sum_{v\in V} f(u,v) = \sum_{v\in V} f(v,u)$ (ausgehend = eingehend)
\end{definition}
\begin{definition}[Wert eines Flusses]
    Der Wert $|f|$ eines Flusses $f: V x V \rightarrow \mathbb{R}$ für ein Flussnetzwerk $G$ ist: \\
    $|f| = \sum_{v\in V} f(s,v) = \sum_{v\in V} f(v,s)$
\end{definition}

\paragraph{Transformationen}\mbox{}\vspace{1em}\\
\includegraphics[width=12cm]{pictures/flussTrans1\IfDarkModeT{_dark}.PNG}\vspace{1em}\\
\includegraphics[width=12cm]{pictures/flussTrans2\IfDarkModeT{_dark}.PNG}

\clearpage
\paragraph{Restkapazitätsgraph}\index{Restkapazitätsgraph}
\begin{itemize}
    \item Wird für Ford-Fulkerson benötigt
\end{itemize}
Restkapazität $c_f(u,v)$: \\
\centerline{$c_f(u,v) = \begin{cases}\hiderowcolors
            c(u,v) - f(u,v) & \text{falls $(u,v) \in E$} \\
            f(v,u)          & \text{falls $(v,u) \in E$} \\
            0               & \text{sonst}
        \end{cases}$}
$G_f = (V, E_f)$ mit $E_f = \{(u,v) \in V x V | c_f(u,v) > 0 \}$
\begin{figure}[h]
    \centering
    \includegraphics[width=12cm]{pictures/restkapazitätsgraph\IfDarkModeT{_dark}.PNG}
    \caption{Beispiel Restkapazitätsgraph}
\end{figure}
Suche eines Pfades von $s$ nach $t$ und Erhöhung aller Flüsse um niedrigsten möglichen Wert auf Pfad



\paragraph{Ford-Fulkerson-Algorithmus}\index{Ford-Fulkerson}\mbox{}
\begin{idea}[Ford-Fulkerson-Algorithmus]\mbox{}
    \begin{itemize}
        \item Suche Pfad von $s$ nach $t$, der noch \fatsf{erweiterbar} ist
        \item Suche dieses Pfades im Restkapazitätsgraphen $G_f$ (mögliche Zu- und Abflüsse)
    \end{itemize}
\end{idea}
Code:
%\begin{noindent}
\begin{codeBlock}[autogobble,escapeinside=||]{title={Ford-Fulkerson(G,s,t,c)}}
    FOREACH e in E do e.flow = 0;
    WHILE there is path p from s to t in |$G_{flow}$| DO
        |$c_{flow}(p)$| = min {|$c_{flow}(u,v)$| : (u,v) in p}
        FOREACH e in p DO
            IF e in E THEN
                e.flow = e.flow + |$c_{flow}(p)$|;
            ELSE
                e.flow = e.flow - |$c_{flow}(p)$|;
\end{codeBlock}
%\end{noindent}
\begin{itemize}
    \item Die Pfadsuche erfolgt z.B. per BFS oder DFS
    \item Laufzeit: $O(|E| \cdot u \cdot |f^*|)$ \\
          ($O(|V| \cdot |E|^2)$ Mit Verbesserung nach Edmonds-Karp) \\
          (wobei $f^*$ maximaler Fluss und Fluss um bis zu $\frac{1}{u}$ pro Iteration wächst)
\end{itemize}
\clearpage
Beispiel:\\
\includegraphics[width=12cm]{pictures/fordbsp1\IfDarkModeT{_dark}.PNG}\\
\includegraphics[width=12cm]{pictures/fordbsp2\IfDarkModeT{_dark}.PNG}
\clearpage
\end{document}
