\documentclass[
    ngerman,
    color=3b,
    dark_mode,
    load_common, % Loads a list of commonly used Packages
    summary,
    boxarc,
    % manual_term,
    % solution=true,
]{rubos-tuda-template} 
% Import all Packages from Main Preamble with relative Path
% \subimport*{../../}{preamble}
% Get Labels from Main Document using the xr-hyper Package
\externaldocument{../../AuD-Zusammenfassung-2020}
% Set Graphics Path, so pictures load correctly
\graphicspath{{../../}}

\begin{document}
\section{NP}\index{NP}
\paragraph{Berechnungsprobleme}\mbox{}\vspace{-2em}\\
\begin{wrapfigure}[1]{r}{.35\textwidth}
      \includegraphics[width=.35\textwidth]{pictures/npmeme.png}
      \caption{NP-Meme}
\end{wrapfigure}
\begin{itemize}
      \item Sind alle Probleme in polynomieller Zeit lösbar? ($O(n^k)$)
      \item Nein $\Rightarrow$ Manche nur in superpolynomieller Zeit lösbar
      \item Polynomielle Probleme: \textit{\string"einfach\string"}
      \item Superpolynomielle Probleme: \textit{\string"hart\string"}
\end{itemize}
\mbox{}\vspace{1cm}

\begin{definition}[Klasse P]\mbox{}
      \begin{itemize}
            \item Klasse aller Polynomialzeitprobleme
            \item Problem ist \textit{effizient lösbar} gdw. es in polynomieller Zeit lösbar ist
            \item Gilt für Polynome beliebigen Grades (auch $n^k$)
            \item Zeitkomplexität $n^k$ mit gro\ss em $k$ bedenklich, jedoch fast nie notwendig
            \item $n$ beschreibt die Länge der Eingabe
      \end{itemize}
\end{definition}
Beispiele: Binäre Addition, Kürzeste Wege, Sortieren, \dots

\begin{definition}[Klasse NP]\mbox{}
      \begin{itemize}
            \item Enthält \textit{\string"einfach zu verifizierende\string"} Probleme (polynomieller Zeit)
            \item Enthält Probleme mit \textit{\string"kurzem Beweis\string"} (Länge polynomiell in Länge der Instanz)
            \item Also: Klasse aller Probleme, deren Lösung in Polynomialzeit verifizierbar ist
      \end{itemize}
\end{definition}
\begin{itemize}

      \item Beispiele: Soduko, 3D-Matching,...
      \item Beispiel: \textit{Faktorisierungsproblem}
            \begin{itemize}
                  \item Jede nicht-Primzahl kann eindeutig als Primzahlprodukt geschrieben werden
                  \item $35 = 5 \cdot 7$, $117 = 3 \cdot 3 \cdot 13$,...
                  \item Faktorsieren auf klassischen Computern schwer
                  \item $n \longrightarrow^{schwer} p,q$
                  \item $n,p,q \longrightarrow^{leicht}$ ist $n = p \cdot q?$
            \end{itemize}
      \item \includegraphics[width=2cm]{pictures/rucksackproblem.png} \textit{Rucksackproblem} auch in polynomieller Laufzeit verifizierbar
\end{itemize}
\clearpage
\paragraph{Hamilton-Kreis-Problem}\mbox{}
\begin{definition}[Hamiltonischer Kreis]\mbox{}
      Zyklus, der alle Knoten, aber nicht unbedingt alle Kanten enthält
\end{definition}
\begin{itemize}
      \item Entscheidungsalgorithmus listet alle möglichen Permutationen der Knoten aus $G$ auf
      \item Prüfung bei jeder Permutation, ob es ein Hamiltonischer Kreis ist
      \item Laufzeit:
            \begin{itemize}
                  \item Kodierung via Adjazenzmatrix: $m$ Knoten $\Rightarrow$ Matrix mit $n = m~x~m$ Einträgen
                  \item $m!$ mögliche Permutationen der Knoten
                  \item $\Omega(m!) = \Omega(\sqrt{n}!) = \Omega(2^{\sqrt{n}})$
                  \item $\Rightarrow$ superpolynomielle Laufzeit (liegt \fatsf{nie} in $O(n^k)$)
            \end{itemize}
      \item \fatsf{Allerdings:} Einfacher, wenn nur Beweis verifiziert werden muss
      \item[] $\Rightarrow$ Test, ob es sich um Permutation der Knoten handelt
      \item[] $\Rightarrow$ Test, ob alle angegebenen Kanten auf Kreis im Graphen existieren
      \item[] $\Rightarrow$ Verifikationsalgorithmus $V$ mit quadratischer Laufzeit
      \item Verifikationsalgorithmus: $V(x,y) = 1/0$ (1, falls Kreis bzw. 0, falls nicht)
      \item Damit: Hamilton-Kreis $\in$ NP
\end{itemize}


\paragraph{Entscheidungsproblem vs Optimierungsproblem}
\begin{itemize}
      \item Optimierungsproblem: Lösung nimmt bestimmten Wert an
      \item Entscheidungsproblem: Binäre Antwort (Ja/Nein)
      \item Bei NP Betrachtung von Entscheidungsproblemen
      \item Optimierungsproblem oft in verwandtes Entscheidungsproblem umwandelbar
      \item Verwandtes Entscheidungsproblem: dem zu optimierenden Wert wird eine Schranke auferlegt
\end{itemize}

\paragraph{P versus NP}\mbox{}\vspace{1em}\\
\includegraphics[width=12cm]{pictures/pnp\IfDarkModeT{_dark}.PNG}
\begin{itemize}
      \item Für viele wichtige Probleme ist jedoch unbekannt, ob sie in P (effizient) lösbar sind
      \item Unbekannt ob $P \neq NP$
      \item Intuitive Frage: Ist das Finden eines Beweises schwieriger als dessen Überprüfung?
      \item[] $\Rightarrow$ Ja, also $P \neq NP$ gilt
      \item In den letzten 50 Jahren kein Beweis für $P = NP$
      \item Eines der wichtigsten offenen Probleme der theoretischen Informatik
      \item Konsequenzen eines Beweises von $P=NP$:
            \begin{itemize}
                  \item $P=NP$: \fatsf{dramatisch}, vieles bisher schwieriges einfacher lösbar (Rucksack, Kryptographie)
                  \item $P\neq NP$: \fatsf{nicht dramatisch}, mgl. interessante Konsequenzen in Kryptographie
            \end{itemize}
\end{itemize}
\clearpage
\paragraph{NP-Vollständigkeit}
\begin{itemize}
      \item Problem befindet sich in NP
      \item Problem ist so \string"schwer\string" wie jedes Problem in NP
      \item Beweis: Zeigen, dass kein effizienter Algorithmus existiert
      \item Werkzeug: Reduktionen (zum Vergleich verschiedener Probleme)
\end{itemize}
\begin{definition}[NP-Härte/NP-Schwere]\mbox{}
      \begin{itemize}
            \item Klassifikation von Problemen als schwierig, trotz fehlender genauer Zuordnung
            \item \textit{Starke Indikatoren}, dass Problem $L$ nicht in P ist:
                  \begin{itemize}
                        \item $L$ ist mindestens so schwierig, wie alle anderen Probleme in NP
                        \item Daraus folgt, dass $L$ nur in P, wenn $P=NP$ (unwahrscheinlich )
                  \end{itemize}
      \end{itemize}
\end{definition}
\begin{definition}[NP-Schwer]
      Problem $L$ ist \fatsf{NP-schwer}, wenn $L' \leq_p L$ für alle $L' \in NP$
\end{definition}
\begin{definition}[NP-Vollständig]
      Problem $L$ ist \fatsf{NP-vollständig}, wenn $L$ sowohl NP-schwer als auch in NP ist\\
      z.B.: Hamilton-Kreis ist NP-vollständig
\end{definition}

\clearpage

\paragraph{Reduktionen}\index{Reduktion}\mbox{}
\begin{idea}[Reduktion]\mbox{}
      \begin{itemize}
            \item Betrachte Problem $A$, das wir in polynomieller Zeit lösen wollen
            \item Bereits bekannt: Problem $B$ (in polynomieller Zeit lösbar)
            \item Benötigt wird Prozedur, die Instanzen der Probleme ineinander überführt
            \item[] $\Rightarrow$ Transformation benötigt polynomielle Zeit
            \item[] $\Rightarrow$ Antworten sind gleich
      \end{itemize}
\end{idea}
\includegraphics[width=15cm]{pictures/reduktion1\IfDarkModeT{_dark}.PNG}\\
\textit{Beispiel:}
\begin{grayInfoBox}

      \begin{itemize}
            \item Intuitiv: Reduktion von A auf B, wenn Umformulierung möglich
            \item[] $\Rightarrow$ Jede Instanz A kann leicht in Instanz von B umformuliert werden
            \item[] $\Rightarrow$ Lösung der Instanz B liefert Lösung von Instanz A
            \item Reduktion: Lösen von linearen Gleichungen auf quadratische Gleichnungen
                  \begin{itemize}
                        \item Lineare Gleichung $ax + b = 0$ $\Rightarrow$ $x= \frac{-b}{a}$
                        \item Quadratische Gleichung $ax^2 + bx + 0 = 0$ $\Rightarrow$ $x = \frac{-b}{a}, x = 0$
                        \item Quadratische Gleichung liefert also auch Lösung für lineare Gleichung
                  \end{itemize}
      \end{itemize}
\end{grayInfoBox}

\textit{Formale Definition:}
\begin{itemize}
      \item[]
            A lässt sich auf B in \fatsf{polynomieller Zeit reduzieren}, mit Schreibweise $A \leq_p B$, wenn eine
            in polynomieller Zeit berechenbare Funktion $f: \{0,1\}^* \rightarrow \{0,1\}^*$ existiert, sodass
            für alle $x \in \{0,1\}^*$ gilt:\begin{align*}
                  x \in A\text{ genau dann, wenn }f(x)\in B
            \end{align*}
      \item[] \includegraphics[width=12cm]{pictures/reduktion2\IfDarkModeT{_dark}.PNG}
\end{itemize}

\clearpage

\paragraph{Travelling-Salesman Problem}\index{Travelling-Salesman Problem}\mbox{}\\
\includegraphics[width=.5\textwidth]{pictures/traveling_salesman\IfDarkModeT{_dark}.png}

\textit{Beschreibung:}
\begin{itemize}
      \item Reisender plant Rundreise durch mehrere Städte
      \item Start und Ziel ist eine vorgegebene Stadt
      \item Jede Stadt nur einmal besuchen
      \item Ziel: Minimale Reisekosten
\end{itemize}
\includegraphics[width=12cm]{pictures/tsp1\IfDarkModeT{_dark}.PNG}\\
\textit{Problem:}
\begin{itemize}
      \item Anzahl der verschiedenen Rundreisen $(n-1)!$
      \item Stark nach oben explodierende Zahlen
      \item Brute-Force für gro\ss{}e $n$ praktisch unmöglich
      \item Es existiert kein effizienter Algorithmus, der das TSP effizient löst
      \item TSP ist \fatsf{NP-vollständig}
\end{itemize}
\textit{Beweis NP-Vollständigkeit}
\begin{itemize}
      \item Zeigen: TSP gehört zu NP und TSP ist NP-schwer
\end{itemize}
\textit{TSP gehört zu NP}
\begin{itemize}
      \item Gegeben: Instanz des Problems TSP, Folge der $n$ Knoten der Tour (Zertifikat)
      \item Verifikationsalgorithmus überprüft, ob Folge jeden Knoten genau einmal enthält
      \item Au\ss{}erdem Aufsummieren der Kantenkosten und überprüfen, ob diese maximal $k$ ist
      \item Verifikation läuft in polynomieller Laufzeit $\Rightarrow$ gehört zu NP
\end{itemize}
\clearpage
\textit{TSP ist NP-schwer}
\begin{itemize}
      \item Wir zeigen $HAM-KREIS \leq_p TSP$
      \item Start: Instanz von $HAM-KREIS$ mit $G=(V,E)$
      \item Konstruiere Instanz von $TSP$
      \item[] $\Rightarrow$ $G'=(V,E')$ mit $E'=\{(i,j):i,j \in V$ und $i\neq j\}$
      \item Definiere Kostenfunktion $c(i,j) = 0$, falls $(i,j) \in E$ / $c(i,j)=1$, falls $(i,j) \notin E$
      \item Instanz von TSP ist $<G', c, 0>$ (Konstruktion in polynomieller Zeit) (0: Kosten von 0)
      \item \fatsf{Zeige jetzt:} $G$ besitzt hamiltonischen Kreis $\Leftrightarrow$ G' enthält Tour mit Kosten $\leq$ 0
      \item $\Rightarrow$ Graph $G$ besitzt einen hamiltonischen Kreis $h$
      \item[] \includegraphics[width=12cm]{pictures/schwer1\IfDarkModeT{_dark}.PNG}
      \item $\Leftarrow$ Graph $G$ besitzt eine Tour $h'$ mit Kosten kleiner gleich 0
      \item[] \includegraphics[width=12cm]{pictures/schwer2\IfDarkModeT{_dark}.PNG}
\end{itemize}
\clearpage
\end{document}